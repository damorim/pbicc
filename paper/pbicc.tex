% This is "sig-alternate.tex" V2.1 April 2013
% This file should be compiled with V2.5 of "sig-alternate.cls" May 2012
%
% This example file demonstrates the use of the 'sig-alternate.cls'
% V2.5 LaTeX2e document class file. It is for those submitting
% articles to ACM Conference Proceedings WHO DO NOT WISH TO
% STRICTLY ADHERE TO THE SIGS (PUBS-BOARD-ENDORSED) STYLE.
% The 'sig-alternate.cls' file will produce a similar-looking,
% albeit, 'tighter' paper resulting in, invariably, fewer pages.
%
% ----------------------------------------------------------------------------------------------------------------
% This .tex file (and associated .cls V2.5) produces:
%       1) The Permission Statement
%       2) The Conference (location) Info information
%       3) The Copyright Line with ACM data
%       4) NO page numbers
%
% as against the acm_proc_article-sp.cls file which
% DOES NOT produce 1) thru' 3) above.
%
% Using 'sig-alternate.cls' you have control, however, from within
% the source .tex file, over both the CopyrightYear
% (defaulted to 200X) and the ACM Copyright Data
% (defaulted to X-XXXXX-XX-X/XX/XX).
% e.g.
% \CopyrightYear{2007} will cause 2007 to appear in the copyright line.
% \crdata{0-12345-67-8/90/12} will cause 0-12345-67-8/90/12 to appear in the copyright line.
%
% ---------------------------------------------------------------------------------------------------------------
% This .tex source is an example which *does* use
% the .bib file (from which the .bbl file % is produced).
% REMEMBER HOWEVER: After having produced the .bbl file,
% and prior to final submission, you *NEED* to 'insert'
% your .bbl file into your source .tex file so as to provide
% ONE 'self-contained' source file.
%
% ================= IF YOU HAVE QUESTIONS =======================
% Questions regarding the SIGS styles, SIGS policies and
% procedures, Conferences etc. should be sent to
% Adrienne Griscti (griscti@acm.org)
%
% Technical questions _only_ to
% Gerald Murray (murray@hq.acm.org)
% ===============================================================
%
% For tracking purposes - this is V2.0 - May 2012

\documentclass{sig-alternate-05-2015}


%% helpers
\newcommand{\ie}{\emph{i.e.}}
\newcommand{\eg}{\emph{e.g.}}
\newcommand{\etal}{\emph{et al.\/}}
\newcommand{\tname}{TNAME}
\newcommand{\pbicc}{PBICC}
\newcommand{\Comment}[1]{}

%% annotations
\newif\ifdraftmode
%% Comment or uncomment the \draftmodetrue line.
\draftmodetrue
\ifdraftmode
 \newcommand{\Fix}[1]{\textbf{[[}{\color{red} #1}\textbf{]]}}
 \newcommand{\marcelo}[1]{\textbf{[[Marcelo: }{\color{magenta} #1}\textbf{]]}}
 \newcommand{\leopoldo}[1]{\textbf{[[Leopoldo: }{\color{blue} #1}\textbf{]]}}
 \newcommand{\paulo}[1]{\textbf{[[Paulo: }{\color{green} #1}\textbf{]]}}
 \newcommand{\note}[1]{\todo[inline,color=red!30,caption={}]{#1}}
\else
 \newcommand{\Fix}[1]{\relax}
 \newcommand{\marcelo}[1]{\relax}
 \newcommand{\leopoldo}[1]{\relax}
 \newcommand{\paulo}[1]{\relax}
 \newcommand{\note}[1]{\relax}
\fi

% For submitted version only.
\pagenumbering{arabic}

% Uncomment this if you need more space
%% \makeatletter
%% \def\@copyrightspace{\enlargethispage{-10pt}\relax}
%% \makeatother

\newcommand{\codesize}{\smaller}
\newcommand{\CodeIn}[1]{\codeid{#1}}
% \|name| or \mathid{name} denotes identifiers and slots in formulas
\def\|#1|{\mathid{#1}}
\newcommand{\mathid}[1]{\ensuremath{\mathit{#1}}}
% \<name> or \codeid{name} denotes computer code identifiers
\def\<#1>{\codeid{#1}}
\newcommand{\codeid}[1]{\ifmmode{\mbox{\codesize\ttfamily{#1}}}\else{\codesize\ttfamily #1}\fi}

%% paragraphs within margins
\usepackage[english]{babel}
\setlength{\emergencystretch}{2pt}

%%%%%%%%%%% packages
\usepackage{todo}
\usepackage[small]{caption}
\usepackage{cite}
\usepackage{subfigure}
\usepackage{listings}
\usepackage{colortbl}
\usepackage{amsmath} 
\usepackage{mathpartir}
\usepackage{fancyvrb}\fvset{fontsize=\small}
\usepackage{xspace}
\usepackage{xcolor}
\usepackage{balance}
\usepackage{wrapfig}
\usepackage{multirow}
%\usepackage{amsfonts}
\usepackage{pifont}
%\usepackage{soul}
%\usepackage{pslatex}
%\usepackage{microtype}
%\usepackage[T1]{fontenc}
%\usepackage{dsfont}
\usepackage{relsize}
\usepackage{tikz}
\usepackage{longtable}
\usepackage{booktabs}
\usepackage{marvosym} 
\usetikzlibrary{matrix,fit,shapes,calc,positioning,shadows,arrows,shapes,backgrounds,decorations.markings,fadings}
\tikzset{
    %Define standard arrow tip
    >=stealth',
    %Define style for boxes
    punkt/.style={
           rectangle,
           rounded corners,
           draw=black, very thick,
           text width=6.5em,
           minimum height=2em,
           text centered},
    % Define arrow style
    pil/.style={
           ->,
           thick,
           shorten <=2pt,
           shorten >=2pt,},
    % call out
    notice/.style  = { draw, rectangle callout, callout relative pointer={#1} }
}
\usepackage{array}
\usepackage{mathtools}
\usepackage{url}
\usepackage{color}

%% Colors
\definecolor{bgBlock}{rgb}{0.22,0.15,0.49}
\definecolor{bgBlockAlert}{rgb}{0.99,0.84,0.31}
\definecolor{fgBlockAlert}{rgb}{0.22,0.15,0.49}
\definecolor{fgBlock}{rgb}{0.99,0.84,0.31}
\definecolor{darkred}{rgb}{0.5,0,0}
\definecolor{darkgreen}{rgb}{0,0.5,0}
\definecolor{darkblue}{rgb}{0,0,0.5}
\usepackage[bookmarks=false]{hyperref}
\hypersetup{ colorlinks,
  linkcolor=darkblue,
  filecolor=darkgreen,
  urlcolor=darkred,
  citecolor=darkblue }


%%%%%%%%%%%%% code listing
\renewcommand{\ttdefault}{pcr}
\lstset{
  basicstyle=\scriptsize\ttfamily,
  keywordstyle=\scriptsize\ttfamily\bfseries,
  language=Java,             % choose the language of the code
  frame=single,              % adds a frame around the code
  aboveskip=0pt,
  belowskip=0pt,
  breaklines=true,           % sets automatic line breaking
  breakatwhitespace=false,   % sets if automatic breaks should only happen at
  showspaces=false,
  %numbersep=5pt,              % Abstand der Nummern zum Text
  %tabsize=2,                  % Groesse von Tabs
  %extendedchars=true,         %
  %breaklines=true,            % Zeilen werden Umgebrochen
}


\begin{document}

% Copyright
\setcopyright{acmcopyright}
%\setcopyright{acmlicensed}
%\setcopyright{rightsretained}
%\setcopyright{usgov}
%\setcopyright{usgovmixed}
%\setcopyright{cagov}
%\setcopyright{cagovmixed}



% DOI
\doi{10.475/123_4}

% ISBN
\isbn{123-4567-24-567/08/06}

%Conference
\conferenceinfo{FSE '16}{Seattle, WA, USA}

\acmPrice{\$15.00}

%
% --- Author Metadata here ---
\conferenceinfo{FSE}{'16 Seattle, USA}
%\CopyrightYear{2007} % Allows default copyright year (20XX) to be over-ridden - IF NEED BE.
%\crdata{0-12345-67-8/90/01}  % Allows default copyright data (0-89791-88-6/97/05) to be over-ridden - IF NEED BE.
% --- End of Author Metadata ---

\title{Pattern-based Inter-Component Communication Discovery\Fix{Analysis?} for
  Android Applications\marcelo{consider changing title to stress
    what it achieves as opposed to how it achieves.}}
%\subtitle{An Empirical Study on Soundiness}
%
% You need the command \numberofauthors to handle the 'placement
% and alignment' of the authors beneath the title.
%
% For aesthetic reasons, we recommend 'three authors at a time'
% i.e. three 'name/affiliation blocks' be placed beneath the title.
%
% NOTE: You are NOT restricted in how many 'rows' of
% "name/affiliations" may appear. We just ask that you restrict
% the number of 'columns' to three.
%
% Because of the available 'opening page real-estate'
% we ask you to refrain from putting more than six authors
% (two rows with three columns) beneath the article title.
% More than six makes the first-page appear very cluttered indeed.
%
% Use the \alignauthor commands to handle the names
% and affiliations for an 'aesthetic maximum' of six authors.
% Add names, affiliations, addresses for
% the seventh etc. author(s) as the argument for the
% \additionalauthors command.
% These 'additional authors' will be output/set for you
% without further effort on your part as the last section in
% the body of your article BEFORE References or any Appendices.

\numberofauthors{3} %  in this sample file, there are a *total*
% of EIGHT authors. SIX appear on the 'first-page' (for formatting
% reasons) and the remaining two appear in the \additionalauthors section.
%
\author{
% You can go ahead and credit any number of authors here,
% e.g. one 'row of three' or two rows (consisting of one row of three
% and a second row of one, two or three).
%
% The command \alignauthor (no curly braces needed) should
% precede each author name, affiliation/snail-mail address and
% e-mail address. Additionally, tag each line of
% affiliation/address with \affaddr, and tag the
% e-mail address with \email.
%
% 1st. author
\alignauthor
Vinicius Souza\\
       \affaddr{Federal University of Pernambuco}\\
       \affaddr{Recife, Brazil}\\
       \email{vcps@cin.ufpe.br}
% 2nd. author
\alignauthor
Leopoldo Teixeira\\
       \affaddr{Federal University of Pernambuco}\\
       \affaddr{Recife, Brazil}\\
       \email{lmt@cin.ufpe.br}
% 3rd. author
\alignauthor Marcelo d'Amorim\\
       \affaddr{Federal University of Pernambuco}\\
       \affaddr{Recife, Brazil}\\
       \email{damorim@cin.ufpe.br}
%\and  % use '\and' if you need 'another row' of author names
%% 4th. author
%\alignauthor Lawrence P. Leipuner\\
%       \affaddr{Brookhaven Laboratories}\\
%       \affaddr{Brookhaven National Lab}\\
%       \affaddr{P.O. Box 5000}\\
%       \email{lleipuner@researchlabs.org}
%% 5th. author
%\alignauthor Sean Fogarty\\
%       \affaddr{NASA Ames Research Center}\\
%       \affaddr{Moffett Field}\\
%       \affaddr{California 94035}\\
%       \email{fogartys@amesres.org}
%% 6th. author
%\alignauthor Charles Palmer\\
%       \affaddr{Palmer Research Laboratories}\\
%       \affaddr{8600 Datapoint Drive}\\
%       \affaddr{San Antonio, Texas 78229}\\
%       \email{cpalmer@prl.com}
}
% There's nothing stopping you putting the seventh, eighth, etc.
% author on the opening page (as the 'third row') but we ask,
% for aesthetic reasons that you place these 'additional authors'
% in the \additional authors block, viz.
%\additionalauthors{Additional authors: John Smith (The Th{\o}rv{\"a}ld Group,
%email: {\texttt{jsmith@affiliation.org}}) and Julius P.~Kumquat
%(The Kumquat Consortium, email: {\texttt{jpkumquat@consortium.net}}).}
%\date{30 July 1999}
% Just remember to make sure that the TOTAL number of authors
% is the number that will appear on the first page PLUS the
% number that will appear in the \additionalauthors section.

\maketitle

\begin{abstract}
\Comment{Android applications are built based on a variety of components. 
Therefore, to analyze such applications, one usually has to identify
inter-component communication (ICC).}
%% PROBLEM
\sloppy{}Inter-Component Communication (ICC) is a common input to a
variety of program analyses for Android\Fix{such as X, Y, and
  Z}.\marcelo{Explain what is an ICC in one sentence.}  \marcelo{Need
  to explain why we care about this
  problem.$\rightarrow$}Unfortunately, existing approaches although
very precise do not scale\marcelo{any other problem? can we be make a
  more-specific statement (e.g., only scale for small apps in terms
  of number of components/activities/code)?}

%% SOLUTION
This paper builds on the observation that ICCs are substantiated
through a small set of patterns and often one ICC
uses\Fix{implements?} more than one of these patterns.  Based on this
observation we propose \tname{}~a pattern-based ICC analysis that is
both lightweight and \Fix{almost as?} precise.

%% EVIDENCE
\marcelo{Don't need to polish this now.}We apply this analysis to
\totalapps{}~applications from the Google Play store, identifying $YY\%$ of
ICC specifications.  Although our analysis does not assure soundness
(no false negatives) or completeness (no false positives) by
definition, our results provide strong evidence that it is competitive
to others in terms of precision and recall. Considering efficiency,
however, our analysis outperforms existing analysis by XXX.  These
results show that our approach can be useful for integration into
other program analysis tools for Android.
\end{abstract}

%
% The code below should be generated by the tool at
% http://dl.acm.org/ccs.cfm
% Please copy and paste the code instead of the example below. 
%
\begin{CCSXML}
<ccs2012>
<concept>
<concept_id>10002944.10011123.10010912</concept_id>
<concept_desc>General and reference~Empirical studies</concept_desc>
<concept_significance>500</concept_significance>
</concept>
<concept>
<concept_id>10011007.10010940.10010992.10010998.10011000</concept_id>
<concept_desc>Software and its engineering~Automated static analysis</concept_desc>
<concept_significance>500</concept_significance>
</concept>
</ccs2012>
\end{CCSXML}

\ccsdesc[500]{General and reference~Empirical studies}
\ccsdesc[500]{Software and its engineering~Automated static analysis}


%
% End generated code
%

%
%  Use this command to print the description
%
\printccsdesc

% We no longer use \terms command
%\terms{Theory}

%\keywords{ACM proceedings; \LaTeX; text tagging}

\section{Introduction (1 page)}

Android marketshare, numbers, importance. 
Briefly explaing that Android applications interact through intents and ICC

Discuss past analysis, how they often are built for soundness, and therefore suffer on scalability. Also mention about trying to capture each and every way of building intents and ICC

Explain that intents and ICC are built following patterns. Use a small
example?\marcelo{Yes, please.  This is very very important.  You can
  have one super-short simple-to-describe example here and have a
  section of example(s) following this intro section.} We can explore this to create lightweight analysis, that are imprecise, but can account for most of the scenarios, as we show in the evaluation (hopefully). 

Brief idea of the solution. 

We make the following contributions:
\begin{itemize}
 \item~\textbf{Idea.} We leverage the idea of overlapping analysis information to
   reduce cost of static analyses.\marcelo{check if this is really
     new.  Highlight how this embraces the philosophy of soundiness.}
 \item~\textbf{Implementation.} We implemented a tool for computing Android ICC in a
   pattern-based way\marcelo{Think about a good name.  Maybe PBICC is
     not the best.  Use a latex macro as placeholder.};
 \item~\textbf{Evaluation.} We perform a study with $XX$ applications, comparing the results with the state-of-the-art analysis for detecting ICC~\cite{epicc,iccta,comdroid}
 \item something about client analysis here? \emph{We analyze the
   impact that our solution has on tools that perform taint analysis
   on Android applications...}\marcelo{I think these 2 bullets should
   be combined.}
\end{itemize}



\section{Motivation (1-2 pages)}

Quickly explain what is ICC\marcelo{remember to define terms before using them.} and give an example of intent and communication between Android components

Go over existing solutions and explain why they are so
expensive\marcelo{I would call this section ``Epicc and IC3'' or ``Existing
  Solutions for ICC Discovery''.  If this is short (say, half a
  column) I would present this in intro very early on, even before
  presenting our approach.}. Is this dependent on the size of the program, or in the number of ICC points?
\begin{itemize}
 \item epicc: build the call graph amounts to large part of the time, \emph{still getting the grasp of the paper, but the problem seems to be that they model the entire application (therefore causing the graph to be too big), while we are mainly interested on specific points of communication}
 \item ic3: builds on top of epicc, including the constraint propagation solver for resolving strings and URIs. Therefore, it can only be as fast as epicc is, but not faster; depending on the intent values it will also spend some time resolving the strings
 \item gator (atanas rountev): still working through the ICSE paper and have also downloaded the tool to figure it out
\end{itemize}

Characterize patterns for creating intents and establishing ICC. Gather the data collected by Vinicius and present here, that in general, half of the ICCs are explicit, and we can infer such information without sophisticated analysis. (\emph{challenge seems to be implicit intents})

\textbf{A explicacao e apresentacao de padroes se daria aqui ou na proxima secao?}

\section{Solution (1-2 pages)}

Explain the idea of patterns using 1-2 concrete examples. (1 from motivation and maybe other here)

The intuition is that patterns can often occur together, therefore we might calculate the intersection and give a precise answer over the ICC specification in most of the program points. 

\subsection{Explicit ICC}

\begin{itemize}
 \item Intent creation with class name (anonymous object)
 \item Use setComponent/setClass on Intent object
\end{itemize}

Both patterns can be specialized when the class name is in a local variable, static class attribute, and so on... Explain this.

\subsection{Implicit ICC}

\begin{itemize}
 \item Intent with get/put extras
 \item Intent with Action, Data, and Category fields
\end{itemize}

\subsection{Dynamic Broadcast Receivers}

Still working on which patterns we can establish for this case.


\subsection{Implementation}

Explain the overall implementation, we collect the information based on the patterns established above. 

Then we have to perform a ``join'' of this information, to make it more precise

This results in a number of ICC specifications, or interfaces. Explain the idea using the 1-2 examples given before.
\section{Evaluation (4 pages)}

\marcelo{I suggest to think about the questions we want to answer
  first: I think it is a good idea to show purpose.  How the technique
  (name) performs in term of precision, recall, and efficiency?  What
  is the practical effect of the technique in a client analysis? (When
  there is loss, how relevant is it?  if not, why?)  The interesting
  question here seems to be: if this is for security, do attackers
  typically care about creating hard-to-find communication patterns?
  Would they care after reading our paper?  Btw., are there client
  analysis other than security (e.g., tainting)?}

\leopoldo{I thought only of taint analysis at first, but I don't know whether we could plug this into other client analysis.}

This section presents the evaluation of \tname{}, our proposed approach for identifying ICC in Android applications. Similar to previous works~\cite{epicc}, we first evaluate our approach in terms of precision of the generated ICC specifications. Moreover, we also compare our results with the results from existing tools, showing how our approach compares in terms of precision, recall, and efficiency. Finally, we investigate the the practical effect of using the technique in a client\leopoldo{To be defined} analysis tool that uses ICC information to compute \emph{something}. \leopoldo{a particular candidate is IccTA, since we have seen that it is relatively easy to integrate}

We have collected \totalapps{}~applications from the Google Play store, extracting the most popular free applications from each category in the store. We performed all of this analysis in a\ldots \leopoldo{detail the environment}

\subsection{Precision of ICC Specifications}

Analysis on precision of the specifications. 

What Epicc did was: for each ICC point in the program \leopoldo{such as startActivity(), for example}, they compute a specification. The idea is that if some startActivity() was called, the specification consists of the values contained in the intent that was passed to this method. They consider a specification ambiguous if there is only one possible value for each of the fields. Therefore, what they measured in terms of precision is whether they generate specifications without ambiguity. \leopoldo{In my opinion, this does not guarantee that the unambiguous specification is correct, but that was the approach they took}

My suggestion is that we adopt the same approach. 

\subsection{Comparison with existing ICC tools}

Compare with Epicc~\cite{epicc}, IC3~\cite{ic3-icse15}, and any other we might find that deals specifically with ICC discovery. 

The idea here would be to run these other tools in the same set of apps and compare the outputs in terms of precision (using the idea of the previous section), efficiency, and also comparing the deltas of our results with theirs so we can discuss what are our strengths/weaknesses compared to them. 

\subsection{Application to Vulnerabilities}

Use IccTA~\cite{iccta} with our ICC information and compare performance and analysis results. 

\section{Related Work (1 page)}

\marcelo{discuss alternative techniques to solve the same or similar
  problem (works related by problem) or use an approach similar in spirit to solve a different
  problem (works related by solution).}

Epicc~\cite{epicc}

IC3~\cite{ic3-icse15}

IccTA~\cite{iccta}

FlowDroid~\cite{flowdroid}

GATOR~\cite{analysis-callbacks-android}

William Enck's papers~\cite{android-security} (grab more)

\section{Conclusions (1/2 page)}



%\begin{table}
%\centering
%\caption{Frequency of Special Characters}
%\begin{tabular}{|c|c|l|} \hline
%Non-English or Math&Frequency&Comments\\ \hline
%\O & 1 in 1,000& For Swedish names\\ \hline
%$\pi$ & 1 in 5& Common in math\\ \hline
%\$ & 4 in 5 & Used in business\\ \hline
%$\Psi^2_1$ & 1 in 40,000& Unexplained usage\\
%\hline\end{tabular}
%\end{table}

%\begin{table*}
%\centering
%\caption{Some Typical Commands}
%\begin{tabular}{|c|c|l|} \hline
%Command&A Number&Comments\\ \hline
%\texttt{{\char'134}alignauthor} & 100& Author alignment\\ \hline
%\texttt{{\char'134}numberofauthors}& 200& Author enumeration\\ \hline
%\texttt{{\char'134}table}& 300 & For tables\\ \hline
%\texttt{{\char'134}table*}& 400& For wider tables\\ \hline\end{tabular}
%\end{table*}
% end the environment with {table*}, NOTE not {table}!



%\begin{figure}
%\centering
%\includegraphics{fly}
%\caption{A sample black and white graphic.}
%\end{figure}
%
%\begin{figure}
%\centering
%\includegraphics[height=1in, width=1in]{fly}
%\caption{A sample black and white graphic
%that has been resized with the \texttt{includegraphics} command.}
%\end{figure}

%\begin{figure*}
%\centering
%\includegraphics{flies}
%\caption{A sample black and white graphic
%that needs to span two columns of text.}
%\end{figure*}
%
%
%\begin{figure}
%\centering
%\includegraphics[height=1in, width=1in]{rosette}
%\caption{A sample black and white graphic that has
%been resized with the \texttt{includegraphics} command.}
%\vskip -6pt
%\end{figure}

%ACKNOWLEDGMENTS are optional
%\section{Acknowledgments}
%This section is optional; it is a location for you
%to acknowledge grants, funding, editing assistance and
%what have you.  In the present case, for example, the
%authors would like to thank Gerald Murray of ACM for
%his help in codifying this \textit{Author's Guide}
%and the \textbf{.cls} and \textbf{.tex} files that it describes.

%
% The following two commands are all you need in the
% initial runs of your .tex file to
% produce the bibliography for the citations in your paper.
\bibliographystyle{abbrv}
\bibliography{pbicc}  % sigproc.bib is the name of the Bibliography in this case
% You must have a proper ".bib" file
%  and remember to run:
% latex bibtex latex latex
% to resolve all references
%
% ACM needs 'a single self-contained file'!
%
%APPENDICES are optional
%\balancecolumns
%\appendix
%%Appendix A
%\section{Headings in Appendices}
%The rules about hierarchical headings discussed above for
%the body of the article are different in the appendices.
%In the \textbf{appendix} environment, the command
%\textbf{section} is used to
%indicate the start of each Appendix, with alphabetic order
%designation (i.e. the first is A, the second B, etc.) and
%a title (if you include one).  So, if you need
%hierarchical structure
%\textit{within} an Appendix, start with \textbf{subsection} as the
%highest level. Here is an outline of the body of this
%document in Appendix-appropriate form:
%\subsection{Introduction}
%\subsection{The Body of the Paper}
%\subsubsection{Type Changes and  Special Characters}
%\subsubsection{Math Equations}
%\paragraph{Inline (In-text) Equations}
%\paragraph{Display Equations}
%\subsubsection{Citations}
%\subsubsection{Tables}
%\subsubsection{Figures}
%\subsubsection{Theorem-like Constructs}
%\subsubsection*{A Caveat for the \TeX\ Expert}
%\subsection{Conclusions}
%\subsection{Acknowledgments}
%\subsection{Additional Authors}
%This section is inserted by \LaTeX; you do not insert it.
%You just add the names and information in the
%\texttt{{\char'134}additionalauthors} command at the start
%of the document.
%\subsection{References}
%Generated by bibtex from your ~.bib file.  Run latex,
%then bibtex, then latex twice (to resolve references)
%to create the ~.bbl file.  Insert that ~.bbl file into
%the .tex source file and comment out
%the command \texttt{{\char'134}thebibliography}.
%% This next section command marks the start of
%% Appendix B, and does not continue the present hierarchy
%\section{More Help for the Hardy}
%The sig-alternate.cls file itself is chock-full of succinct
%and helpful comments.  If you consider yourself a moderately
%experienced to expert user of \LaTeX, you may find reading
%it useful but please remember not to change it.
%%\balancecolumns % GM June 2007
%% That's all folks!
\end{document}
